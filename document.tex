\documentclass{beamer}
\usepackage[polish]{babel}
\usepackage[utf8]{inputenc}
\usepackage[T1]{fontenc}

\usetheme{Berlin}
\usecolortheme{seahorse}

\title{Prezentacja o psach}
\author{Gosia Jasińska}
\institute{UG}

\begin{document}
	\maketitle
	\begin{frame}{Pieski}{wstęp}
		Pieski to zwierzęta domowe hodowane m.in. dla przyjemności lub do polowaniań.
		Są bardzo słotkie, zabawne i urocze, ale też odważne, oddane i waleczne.
		Kto nie lubi tych uroczych czworonogów?
		Warto zatem poświęcic im chwilę uwagi.
	\end{frame}
	\begin{frame}{Powody dla których warto mieć psa:}
		\begin{itemize}
			\item psy są lojalnymi przyjaciółmi \pause
			\item psy pomagaja radzić sobie ze stresem \pause
			\item właściciele psów mają mniejsze dolegliwości zdrowotne \pause
			\item głaskanie psów obniża ciśnienie krwi \pause
			\item psy uczą dyscypliny i odpowiedzialności
		\end{itemize}
	\end{frame}
	\begin{frame}{Wiek}{Ile tak właściwie żyją psy?}
		Wiek psa określa się m.in. po oględzinach stanu uzębienia oraz ilości siwych włosów na głowie.   
		   
		Według Kazimierza Ściesińskiego zestawienie porównawcze wieku psa w stosunku do człowieka przedstawia się następująco: 
		
		.
		
			\begin{tabular}{c| |c|c|c|c|c|c}
				\textbf{pies} & 6 msc & 12 msc & 2 lata & 4 lata & ... & 16 lat \\
				\hline
				\textbf{człowiek} & 10 lat & 15 lat & 24 lata & 32 lata & ... & 80 lat \\
			\end{tabular}
		
		.
		
		Najstarszym psem w historii, według Księgi rekordów Guinnessa, był Bluey, który przeżył \textbf{29 lat, 5 miesięcy i 7 dni}.
	\end{frame}
		\begin{frame}{TOP 5 ras psów}{moja opinia}
		\begin{enumerate}
			\item Chow Chow \pause
			\item Golden Retriver \pause
			\item Corgi \pause
			\item Bernardyn \pause
			\item Mops
		\end{enumerate}
	\end{frame}
	\begin{frame}{Chow Chow}
		\begin{figure}
			\includegraphics[width=0.8\linewidth]{chow chow}
		\end{figure}
	\end{frame}
	\begin{frame}{Golden Retriver}
		\begin{figure}
			\includegraphics[width=0.8\linewidth]{golden}
		\end{figure}
	\end{frame}
	\begin{frame}{Corgi}
		\begin{figure}
			\includegraphics[width=0.8\linewidth]{corgi}
		\end{figure}
	\end{frame}
	\begin{frame}{Bernardyn}
		\begin{figure}
			\includegraphics[width=0.8\linewidth]{bernardyn}
		\end{figure}
	\end{frame}
	\begin{frame}{Mops}
		\begin{figure}
			\includegraphics[width=0.8\linewidth]{mops}
		\end{figure}
	\end{frame}
	\begin{frame}{Znane psy}
		Cenione i znane psy, które zachwyciły świat:
		\begin{itemize}
			\item Lassie \pause
			\item Łajka \pause
			\item Szarik \pause
			\item Balto
		\end{itemize}
	\end{frame}
	\begin{frame}{Lassie}{suczka owczarka szkockiego collie}
		\begin{columns}
		\begin{column}{0.5\textwidth}
			Jest bohaterką powieści “Lassie, wróć” autorstwa Erica Knighta. Lessie łączy wielka przyjaźń z chłopcem imieniem Joe. Niestety jego rodzice, będąc w trudnej sytuacji finansowej, zmuszeni są sprzedać ukochanego psiaka ich syna. W rezultacie Lessie, przepełniona tęsknotą, ucieka od swojego nowego właściciela i przemierzając wiele kilometrów, wraca do swojego przyjaciela Joe.
		\end{column}
		\begin{column}{0.5\textwidth}
			\includegraphics[scale=0.25]{lassie.jpg}
		\end{column}
		\end{columns}
	\end{frame}
	\begin{frame}{Łajka}{suczka mieszanej rasy}
		\begin{columns}
			\begin{column}{0.5\textwidth}
				\includegraphics[scale=0.75]{lajka.jpg}
			\end{column}
			\begin{column}{0.5\textwidth}
				Łajka to pierwszy pies w kosmosie. Co więcej jest to psiak, który odbył pozaziemską podróż w kosmos jeszcze przed człowiekiem. Ta psia astronautka wyruszyła w swoją podróż 3 listopada 1957 na pokładzie radzieckiej satelity Sputnik 2.
			\end{column}
		\end{columns}
	\end{frame}
	\begin{frame}{Szarik}{samiec rasy owczarek niemiecki}
		\begin{columns}
			\begin{column}{0.5\textwidth}
				Szarik to jeden z głównych, tytułowych bohaterów kultowego serialu “Czterej pancerni i pies”. Ten bohaterski owczarek niemiecki szybko zyskał sympatię widzów stając się ulubieńcem publiczności. Szarik to psi żołnierz, który jest częścią czteroosobowej załogi czołgu Rudy, stacjonującego podczas II Wojny Światowej.
			\end{column}
			\begin{column}{0.5\textwidth}
				\includegraphics[scale=0.32]{szarik.jpg}
			\end{column}
		\end{columns}
	\end{frame}
	\begin{frame}{Balto}{samiec rasy husky}
		\begin{columns}
			\begin{column}{0.5\textwidth}
				\includegraphics[scale=0.15]{balto.jpg}
			\end{column}
			\begin{column}{0.5\textwidth}
				Balto to psiak , który w 1925 uratował życie dwanaściorgu chorych dzieci.  Jedynym ratunkiem dla nich były szczepionki znajdujące się tylko i wyłącznie w szpitalu oddalonym o 1600 km od rodzinnego miasta Balto – Nome. Dodatkowo panujące wówczas bardzo ciężkie warunki atmosferyczne uniemożliwiały podróż. Jedynym wyjściem z tej sytuacji był transport psim zaprzęgiem.
			\end{column}
		\end{columns}
	\end{frame}
	\begin{frame}{Bibliografia}{dziękuję za uwagę!}
		\begin{columns}
			\begin{column}{0.5\textwidth}
			Źródła:
			
			\href{https://pl.wikipedia.org/wiki/Pies_domowy}{\beamergotobutton{wikipedia.org}}	
			\href{https://www.bowlandbone.pl/historie-cenionych-i-znanych-psow-ktore-zachwycily-swiat/}{\beamergotobutton{bowlandbone.pl}}
			
			google grafika
			\end{column}
			\begin{column}{0.5\textwidth}
				\includegraphics[scale=0.15]{pies.jpg}	
			\end{column}
		\end{columns}
	\end{frame}
	
\end{document}